
\documentclass{standalone}
\usepackage[oldvoltagedirection]{circuitikz}
\usepackage{amsmath}

\begin{document}
\begin{circuitikz}[american voltages]
  % outer loop
  \draw
  node[ocirc] (A) at (0,4) {}
  node[ocirc] (B) at (4,4) {}
  node[ocirc] (C) at (8,4) {}
  node[ocirc] (D) at (4,0) {}
  (D) to[short, *-] (0,0)
  (0,0) to [V, v=$50\,\mathrm{V}$] (A)
  (A) to[short, *-] (0,8)
  (0,8) to [R, l=$1\,\Omega$] (8,8)
  (8,8) to[short, -*] (C)
  (C) to[cV, v=$15\,i_\phi$, invert] (8,0)
  (8,0) to[short, -*] (D)
  % 5 Ohm resistor to B node
  (A) to[R, -*, l=$5\,\Omega$] (B)
  % B node to D node through 20 Ohm resistor with i_phi current
  (B) to[R, l=$20\,\Omega$, i>^=$i_\phi$] (D)
  % 4 Ohm resitor from B to C
  (B) to[R, l=$4\,\Omega$] (C)
  % Identificador da Malha 1
  % circulo
  [->, thick] (1.5, 2) arc[start angle = 180, end angle = -90, radius = 0.5]
  % rotulo
  node [] (M_1) at (2, 2) {$M_1$}
  % corrente da Malha
  node [] (i_M_1) at (2, 1) {$\overset{\leftarrow}{i_1}$};
  % Identificador da Malha 2
  % circulo
  \draw 
  [->, thick] (3.5, 6) arc[start angle = 180, end angle = -90, radius = 0.5]
  % rotulo
  node [] (M_2) at (4, 6) {$M_2$}
  % corrente da Malha
  node [] (i_M_2) at (4, 5) {$\overset{\leftarrow}{i_2}$};
  % Identificador da Malha 3
  % circulo
  \draw 
  [->, thick] (5.5, 2) arc[start angle = 180, end angle = -90, radius = 0.5]
  % rotulo
  node [] (M_3) at (6, 2) {$M_3$}
  % corrente da Malha
  node [] (i_M_3) at (6, 1) {$\overset{\leftarrow}{i_3}$};
\end{circuitikz}
\end{document}
